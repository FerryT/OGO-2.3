\section{Validatie van aannamen}
Voordat we de aannamen kunnen valideren, moeten we deze aannamen natuurlijk eerst identificeren. De aannamen zijn al uitgebreid besproken in het document \emph{Het ontwerp van het spel}. We zullen hier de belangrijkste punten herhalen:
\begin{enumerate}
\item[(i)] Er kunnen elke seconde 300 berichten over het netwerk worden gestuurd. Deze aanname is natuurlijk alleen realistisch als we nog de grootte van de berichten beperken. Alle gebruikte berichten in het spel zijn zeer klein. Daarom beperken we de grootte van de berichten tot 100 bytes, wat ruim voldoende is voor alle besproken berichten in het ontwerp. Hierbij is de grootte exclusief eventuele overhead van bijvoorbeeld TCP of IP.
\item[(ii)] Verder hebben we aangenomen dat de RTT, oftewel de \emph{Round-Trip Time}, kleiner is dan 200 ms. Dit betekent dus dat het niet langer duurt dan 100 ms om een bericht van speler A naar speler B te sturen.
\end{enumerate}

We hebben deze aannamen uitgebreid getest. Hiertoe hebben we een apart programma geschreven. In deze test stuurden twee computers om en om 10 berichten naar elkaar van 50 bytes. Deze test hebben we drie keer uitgevoerd over \textsc{lan}. We gebruikten hiervoor \textsc{Ethernet} van 100 Mbps. In de eerste test konden we 1568 berichten per seconde sturen, in de tweede test 1323 berichten per seconde en in de derde test 1459 berichten per seconde.

Vervolgens hebben we de RTT onder dezelfde omstandigheden getest. Hierbij hebben we TCP gebruikt om de RTT te meten. In deze test stuurden we een totaal van 100 berichten van 10 bytes per stuk gelijkmatig verdeeld over 10 seconden. Ook deze test hebben we drie keer uitgevoerd. De gemiddelde RTT in de eerste test was 13.5 milliseconde, in de tweede test 12.8 milliseconde en in de laatste test 16.1 milliseconde.
