\section{Klassediagram}

We zullen nu het klassediagram op een \emph{top-down} manier bespreken. Een aantal basisklassen, zoals Point2d, Point3d en Vector3d,  zijn hierin voor de overzichtelijkheid niet aangegeven. We maken bovendien de afspraak dat de types Percentage/Power/Time/Duration vrij kunnen worden gebruikt. Deze kunnen bijvoorbeeld reals of integers zijn, afhankelijk van de implementatie.  We zijn nu klaar om de gebruikte interfaces te bespreken.

\subsection{Interfaces}
Er zijn twee interfaces in het klassediagram: de zogenaamde Object interface en BoundingObject interface. We geven het implementeren van een interface aan door <<interface naam>> voor de naam van de klasse, die de interface implementeert.  De klasse Object bevat een aantal standaard attributen om de conversie tussen globale en lokale co\"ordinaten te bewerkstelligen. Dit zijn origin, yaw, patch en roll. Hier zijn de hoeken yaw, pitch en roll in graden tussen 0 en 360. 0 is in dit geval inclusief, 360 is exclusief. 

Bovendien bevat Object nog het attribuut children, dit is een array van Objects. De reden hiervoor is dat dit hi\"erarchisch modelleren mogelijk maakt. Vervolgens bevat Object nog een aantal methoden: draw, render, prerender en postrender. De prerender methode zorgt voor de transformatie van globale co\"ordinaten naar lokale co\"ordinaten. De postrender methode zorgt voor de transformatie van lokale co\"ordinaten naar globale co\"ordinaten. 

De draw methode zorgt voor het tekenen zelf. Deze methode is natuurlijk abstract gemaakt. De render methode combineert alle voorgaande methoden: het roept eerst prerender aan. Vervolgens wordt draw aangeroepen. Daarna wordt ook de draw functie van de Objects in de children array aangeroepen. Als laatste wordt nog postrender aangeroepen.

De interface BoundingObject is een subklasse van Object. De aard van deze type objecten is zodanig dat ze begrensd zijn. Hier hoort dus een corresponderende \emph{bounding box} bij. De bounding box staat in het attribuut bBox van het type BoundingBox. Ook in dit geval geldt dat de klasse BoundingBox niet in het klassediagram is opgenomen voor de overzichtelijkheid. 

De bounding box is niet alleen handig voor het tekenen. Het is namelijk ook mogelijk met deze bounding box te testen voor \emph{collisions}. Hier wordt getest of er een collision plaatsvindt van het object met de lijn gegeven door het startpunt origin en met richting direction. Het object zelf of een van de children wordt teruggeven in het geval van een collision. Als er geen collision is, dan zullen we \emph{null} teruggeven. Dit kan bijvoorbeeld gebruikt worden bij het schieten.

\subsection{Model van de sc\`ene}
We zijn nu klaar om de klassen, die worden gebruikt om de sc\`ene te modelleren, te bespreken.  Al deze klassen implementeren de interface BoundedObject. Merk op dat de klasse Object dus niet gebruikt wordt, behalve dat BoundedObject een subklasse hiervan is. Deze klasse kan echter nuttig zijn voor verdere uitbreiding van het spel. We beginnen met de World klasse. Deze bevat een attribuut homePlayer van het type Player. Dit attribuut wordt gebruikt om de speler zelf te identificeren. 

De walkHomePlayer wordt gebruikt door de speler om rond te lopen door de wereld. Er wordt dan periodiek een bericht naar alle andere spelers gestuurd om dit door te geven. Verder wordt de walk functie van Player aangeroepen, deze functie wordt verderop nog besproken. De walkPlayer in World wordt gebruikt om andere spelers door de wereld rond te laten lopen. De yesno boolean wordt gebruikt om aan te geven of de speler moet lopen of stoppen met lopen. De shootHomePlayer en shootPlayer hebben een vergelijkbare interpretatie.

De buildBuilding wordt gebruikt door de speler om een nieuw gebouw neer te zetten. Hiervoor moet natuurlijk een punt worden aangegeven op de grond en het gewenste gebouw. Het gewenste gebouw is dan van het type Building. Als laatste is er nog de functie pickup. Deze wordt gebruikt als een speler een object, zoals een muntje, opraapt. Merk op dat dit niet noodzakelijk de speler zelf hoeft te zijn. Opraapbare objecten zijn van het type Droppable. 

\subsubsection{Het terrein}
World heeft een Terrain, die het terrein van de sc\'ene voorstelt. Het terrein wordt gemodelleerd door een grid. De methode groundCollision wordt gebruikt bij het bouwen van een gebouw. In deze methode wordt berekend welke cel van de grid is aangeklikt. Merk op dat de cellen van de grid worden aangeklikt met behulp van het vizier en dat het vizier altijd in de huidige kijkrichting staat. Voor groundCollision kan dus de huidige locatie en kijkrichting worden gebruikt. 

De methode getHeight geeft de hoogte van het terrein voor een zekere cel terug. Op dit moment is ons terrein in principe nog vlak, maar dit willen we later nog uitbreiden. Als we het terrein hoogte willen geven, is een mogelijke optie om dit in de grid op te slaan. Dit staat op dit moment nog niet in het klassendiagram: het is immers ook mogelijk om deze elke keer te berekenen bij eenvoudige terreinen.

Terrain heeft een array van Droppables en een matrix van Structures. Droppable stelt een opraapbaar object voor: dat is nu alleen nog een muntje. Een opraapbaar object heeft een zekere waarde. Bovendien zal een opraapbaar object na een zekere tijd verdwijnen, hiervoor is het attribuut dieTime van het type Time. Als laatste is er nog de methode onPickup.
% wat doet, zie ook World?

De klasse Structure is abstract: de klassen Mine en Building zijn de subklassen. In de klasse Mine wordt opgeslagen hoeveel inkomen deze mijn genereert in een zekere periode. Deze periode is vast voor alle mijnen. Hiervoor kunnen we dus een constante defini\"eren. De klasse Building is ook weer abstract. Een Building heeft altijd \'e\'en corresponderend team. Het heeft de attributen cost, income, buildTime, buildDuration, attackPower en health.

Het attribuut cost staat voor de totale kosten van het gebouw. Income representeert het inkomen dat het gebouw genereert. Ook hier geldt dat de tijdseenheid vast is. BuildTime representeert het moment waarop de speler de opdracht heeft gegeven het gebouw neer te zetten, BuildDuration vertelt hoelang het bouwen duurt. Dit wordt gebruikt door de animatie van het gebouw. Het wordt verder gebruikt om te bepalen wanneer het neerzetten van het gebouw voltooid is. 

AttackPower geeft aan hoe ernstig het harnas wordt beschadigd, als het gebouw een speler neerschiet. Health geeft de sterkte van het gebouw aan: als de health nul wordt, gaat het gebouw kapot. We merken op dat de attributen income en attackPower geen betekenis hebben voor alle gebouwen. Zo heeft attackPower geen betekenis voor de resourceMine. Als dit het geval is, zal altijd de waarde 0 moeten worden gekozen voor dit attribuut.

De klasse ResourceMine, DefenseTower en HQ zijn subklassen van Building. Een ResourceMine is het gebouw dat over een Mine kan worden geplaatst. Mine staat dus voor de delfplaats en ResourceMine voor de mijn. Een Mine is dus in principe hoeft dus niet van een team te zijn, terwijl een ResourceMine dat altijd zal zijn.

\subsubsection{De speler en teams}
World heeft in principe twee Teams, deze klasse representeert de verschillende teams. Later kunnen we dit nog uitbreiden naar meer dan twee teams. Elk team heeft een zekere hoeveelheid goud in de kas. Dit wordt opgeslagen in het attribuut resources. Bovendien heeft Team een array van Players. Hierin staan natuurlijk de spelers van dat team.

De klasse Player modelleert een speler in het spel. In het attribuut health wordt de sterkte van het harnas opgeslagen. Het attribuut lastShoot representeert het laatste moment, waarop de speler heeft geschoten. Hierdoor kunnen we het spel uitbreiden, zodat de speler pas weer kan schieten na een zeker \emph{cooldown} periode. De methode fireLaser verzorgt de animatie bij het schieten. Hiervoor wordt de klasse Laserbeam gebruikt: deze klasse heeft een fireTime en fireDuration. De fireTime is het moment van vuren. De fireDuration bepaalt dan hoe lang de animatie van het schot duurt. De methode walk in Player zorgt voor de animatie als de speler loopt.
% waarvoor is connection nodig, zit dat niet in communicator?

\subsubsection{De camera}
Als laatste heeft de World nog een Camera. We laten het nog open voor de implementatie welke attributen worden gebruikt om de camera te representeren. De klasse PlayerCamera is een subklasse van Camera.
% Camera vs PlayerCamera, Wereld heeft Camera?

\subsection{Statische objecten}
Er is nog \'e\'en cruciaal statisch object voor de implementatie van het spel, waarvan de functies door alle andere klassen kunnen worden gebruikt. Dit is de klasse Communicator, die de communicatie met andere spelers verzorgt.  Voor een uitgebreide beschrijving van het protocol, zie INSERT REFERENCE. Met een boolean in deze klasse wordt aangegeven of dit proces de token heeft. Verder wordt opgeslagen wanneer de token voor het laatst wordt ontvangen. Hiervoor gebruiken we het type Date, die de huidige tijd representeert.
% INSERT REFERENCE
% methode uitleg van Communicator, ging veranderen toch?

De klasse Communicator maakt gebruik van de klasse Message. De klasse Message representeert het bericht dat moet worden verzonden. Het heeft een attribuut type, die het soort bericht aangeeft. Dit attribuut is van het type MessageType, dat voor de overzichtelijkheid is weggelaten uit het klassediagram. MessageType is immers een simpele enumeratie.

Verder heeft Message nog een boolean requiresToken, die aangeeft of de token vereist is om dit soort bericht te mogen versturen.  Als laatste heeft het message nog een array van argumenten. Het type Argument hangt af van de implementatie, hiervoor kan bijvoorbeeld string of byte worden gekozen. Verder zijn er nog de methoden addInteger, addReal, addString, addIntegers en addReals. Deze methoden voegen de type in de naam van de methode toe aan het bericht.