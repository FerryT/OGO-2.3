IN GAME PROTOCOL

\subsection{Het spel}
De volgende berichten worden gebruikt voor communicatie tijdens het spel. Deze berichten worden dan naar alle andere spelers gestuurd. Sommige berichten zijn hierbij aangegeven met een R. Deze berichten krijgen een speciale behandeling, zoals wordt besproken in INSERT REFERENCE. Als eerste is er een zogenaamd \emph{GameChat} bericht. Dit bericht vervult een vergelijkbare taak als het Chat bericht in de lobby. Het GameChat bericht heeft de volgende syntax:
\bericht{GAMECHAT, message.}

Een speler zal periodiek zijn huidige positie doorsturen. Bovendien wordt hierbij de huidige richting meegegeven. Hiervoor wordt het \emph{Move} bericht verstuurd:
\bericht{MOVE, x, y, z, vx, vy, vz.}

Als een speler heeft geschoten, dan wordt dit naar alle andere spelers gestuurd. Hierbij wordt de oorsprong van het schot en de richting van het schot meegestuurd. Merk op dat deze overeenkomen met respectievelijk de positie van de speler en de kijkrichting. De bedoeling van dit bericht is enkel om de bijbehorende animatie weer te geven: eventuele schade aan het harnas wordt apart afgehandeld.
\bericht{FIRE, x, y, z, vx, vy, vz.}

De spelers houden lokaal de totale hoeveelheid goud van het team bij. Spelers zullen deze waarde periodiek versturen. Als deze waarden tussen verschillende spelers van hetzelfde team niet overeenkomen, zullen we deze waarden middelen. Het bericht \emph{Team} verstuurt de ID van het team en de hoeveelheid goud:
\bericht{TEAM, tid, teamGold.}

Verder is er nog een \emph{Respawn} bericht. Dit bericht wordt verstuurd nadat een speler is dood gegaan. In dit bericht wordt de nieuwe locatie meegestuurd, dit zal in principe het commandocentrum van het bijbehorende team zijn:
\bericht{R RESPAWN, x, y, z.}

Na een schot zal een speler berekenen of een speler van het andere team is geraakt. Als dit het geval is, dan zal de geraakte speler en de hoeveelheid schade worden verstuurd. Merk op dat dit lokaal wordt berekend. Het hoeft niet noodzakelijk te zijn dat het schot van de speler zelf komt, het kan ook komen van een gebouw in het bezit van een speler. De geraakte speler en de hoeveelheid schade wordt doorgegeven in het bericht \emph{Hit}:
\bericht{R HIT, pid, damage.}

Als een speler is doodgegaan, zal hij dit kenbaar maken aan alle andere spelers. Dit wordt gedaan door het \emph{Died} bericht, waarin ook de dader is opgenomen. Op zich is de informatie over de dader irrelevant voor het spel, maar dit kan wel worden gebruikt om het spel verder uit te breiden.
\bericht{R DIED, pid.}

Een speler zal een voorwerp achterlaten bij het doodgaan. In het huidige ontwerp zal dit altijd een muntje zijn. Dit muntje krijgt meteen ook een unieke ID en een zekere waarde. Verder zal nog de lokatie van het muntje worden meegestuurd. Als laatste zal nog het type voorwerp worden meegestuurd, dit argument kan worden gebruikt voor verdere uitbreiding. Samen geeft dit het \emph{Drop} bericht:
\bericht{R DROP, id, x, y, z, value, type.}

Als een speler een voorwerp opraapt, zal er een \emph{Take} bericht worden verstuurd. Hierbij zal natuurlijk ook de ID van het opgeraapte voorwerp worden meegestuurd.
\bericht{R TAKE, id.}

Een speler kan natuurlijk ook een gebouw bouwen. Het spel heeft de lokale verantwoordelijkheid dat de bouwplaats toegestaan is. In het \emph{Build} bericht wordt naast de lokatie ook nog een uniek ID en het type van het gebouw meegestuurd:
\bericht{R BUILD, id, x, y, z, type.}

Als een speler een gebouw heeft geraakt met een schot, dan zal de ID van het gebouw worden verstuurd en de hoeveelheid schade. Dit wordt gedaan in het \emph{Attack} bericht:
\bericht{R ATTACK, id, damage.}

Op een gegeven moment kan het ook gebeuren dat het gebouw is vernietigd. De eigenaar van het gebouw heeft de verantwoordelijkheid om dit te controleren. Als het gebouw is vernietigd, dan wordt de ID van het gebouw verstuurd. Bovendien zal de dader hierbij worden verstuurd net zoals in het geval dat een speler is doodgegaan. Hiertoe wordt het \emph{Destroy} bericht gebruikt:
\bericht{R DESTROY, id, pid.}

Als laatste is er nog het \emph{End} bericht. Dit bericht wordt verstuurd zodra een team heeft gewonnen: oftewel als het commandocentrum van het andere team is vernietigd. In dit bericht wordt het winnende team meegestuurd:
\bericht{R END, tid.}