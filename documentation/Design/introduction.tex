Er werd ons gevraagd om een interactief, gedistribueerd 3D spel te ontwikkelen. Het 3D spel mag geen leider of server aanwijzen, dat willen zeggen dat elke speler dezelfde code moet uitvoeren en alle spelers op gelijke voet staan. De enige uitzondering is tijdens het opzetten van het spel. Ons spel, \textsc{GOTO}, wordt beschreven in het document \emph{Informele Specificatie}. Het idee van het spel is een combinatie van een \emph{shiet-spel} en een \emph{strategie-spel}. Hierbij kunnen gebouwen worden geplaatst, bovendien heeft elke speler een laser om mee te schieten. Elke speler zit in een team samen met andere spelers. Voor het volledige concept verwijzen wij naar \emph{Informele Specificatie}.

Om het spel te kunnen implementeren maken we eerst een ontwerp voor de structuur. Dit wordt gedaan door het programma op te delen in meerdere (lokale) onderdelen. Bij dit opdelen in meerdere onderdelen willen we de onderdelen zo onafhankelijk mogelijk maken: immers, dan kunnen ze ook onafhankelijk worden ontwikkeld.

Nadat het programma is opgedeeld in meerdere onderdelen, zullen we elk onderdeel precies beschrijven door een \textsc{uml}-diagram en bijbehorende toelichting. We zullen uitleggen wat elke functie doet en waar de functie gebruik van maakt. Nadat elk onderdeel precies is beschreven, beschrijven we de samenhang tussen de verschillende onderdelen. We geven aan hoe elk onderdeel communiceert. Hier zijn er verschillende opties: door variabelen aan te passen, methoden aan te roepen om de staat van een onderdeel te veranderen of door een methode aan te roepen welke het onderdeel informeert dat er een taak uitgevoerd moet worden. We zullen door middel van een \textsc{msc} de samenhang proberen te verduidelijken.

Het programma moet ook communiceren met de buitenwereld, daarom geven we ook aan hoe de lokale onderdelen communiceren met de buitenwereld. Dit wordt gedaan door het protocol te beschrijven. Het protocol wordt ook opgedeeld in meerdere onderdelen. Elk onderdeel correspondeert met een onderdeel van het spel: dat wil zeggen een onderdeel voor het vinden van medespelers, een onderdeel voor het opstarten van het spel en een onderdeel voor de communicatie tijdens het spel.
