\documentclass[a4paper,11pt]{article}
\usepackage{graphicx,listings,float,geometry, amsmath}
%\usepackage[firstpage]{draftwatermark}
%\SetWatermarkLightness{0.5}
%\SetWatermarkScale{4}
\setcounter{tocdepth}{2}
\usepackage[dutch]{babel}
\newcommand{\protoref}{PROTOREF!}

\geometry{
	includeheadfoot,
	margin=2.54cm
}

\begin{document}
	\begin{titlepage}
	\begin{center}
		
		{\Huge Informele Specificatie \\[0.5cm]OGO 2.3 - Multiplayer Game}\\[0.5cm]
		\rule{\linewidth}{0.5mm}\\[0.5cm]
				\bigskip
		\huge \textit{``Grudge of the Oblivious''}
		
		{\Large
		Luca van Ballegooijen, Tim van Dalen, \\
		Carl van Dueren den Hollander, Peter Koymans,\\
		Kay Lukas en Ferry Timmers\\[1cm]
		}
		
		{\large
		OGO 2.3\\
		Groep 3 \\[1cm]
		Faculteit Wiskunde en Informatica\\
		Technische Universiteit Eindhoven\\[1cm]
		}
		
		\begin{abstract}

    In dit document zullen we de kritieke punten bij dit project identificeren. Vervolgens bekijken we de taken die bij dit project een rol spelen. Tenslotte zullen we dit gebruiken om een werkplan voor het project op te stellen.
\end{abstract}


		\vfill

		{\large \today}
	\end{center}
\end{titlepage}

	
	\tableofcontents
	\newpage

	\section{Introductie}
	Informatici in de hedendaagse wereld komen vaak in aanraking met het ontwerpen van complexe programma's. Complexe programma's hebben vaak ook een netwerk aspect en een grafische aspect. Zelfs bij technische programma's, die vaak minder grafisch intensief zijn, als matlab worden al deze aspecten verenigd. Dit drukt de noodzaak uit dat elke informaticus basiskennis heeft van computergrafiek, computernetwerken en het ontwerpen van complexe programma's. Al deze aspecten worden verenigd in dit project: \textsc{ogo} 2.3 - Nethunt. Dit document ligt ... INSERT MORE

Voor dit project werd ons gevraagd om een interactief, gedistribueerd 3D-spel te ontwikkelen. Dit houdt in dat elk speler lokaal dezelfde spelsituatie heeft, en deze op het scherm van de speler worden afgebeeld. Natuurlijk kan de afbeelding op het scherm verschillen per speler. Een andere eis was dat er voedsel aanwezig moet zijn. E\'en van de randvoorwaarden was dat elke speler ... INSERT MORE
	\newpage
    
    
    \section{Beschrijving onderdelen}
    Wij verdelen de onderdelen op in drie zoveel mogelijk onafhankelijke onderdelen. Hierbij onderscheiden wij:
    \begin{itemize}
    	\item Een onderdeel die het communiceren naar andere speler mogelijk maakt;
	\item Het model van de sc\`ene. Denk hierbij aan de staat, locatie en grafische modellen van gebouwen, spelers en het terrein;
	\item De gebruikersomgeving die de interactie van de speler met het spel mogelijk maakt en ook het grafische model weergeeft tijdens het spel.
    \end{itemize}
    
    Als uitgangspunt ontwerpen wij de verschillende onderdelen dat elk van de onderdelen zoveel mogelijk zelfstandig ontwikkeld kunnen worden.
    	
    \subsection{Protocol module}
    	De protocol module is het onderdeel die het mogelijk maakt om:
	\begin{itemize}
		\item Een nieuw spel, beginnend in de opstart fase, op te starten;
		\item De lijst van alle spellen, welke in de opstart fase, in het subnet van de computer weer te geven;
		\item Mee te doen met bestaand spel die in de opstart fase zit.
		\item Het spel te beginnen;
		\item Mutaties van het spel te ontvangen of te versturen.
	\end{itemize}
	De manier waarop deze communicatie verloopt is beschreven in \protoref. Zodra de protocol module opgestart is, zal de module zijn taken uitvoeren, zonder actieve noodzaak om functies aan te roepen. Dit betekent, dat het protocol uit zichzelf berichten kan ontvangen en verwerken. Door middel van threads wordt er voor gezorgd dat de protocol de taken onafhankelijk kan uitvoeren.
	
    \subsection{Model van de sc\`ene}
    Het model van de sc\`ene houd de huidige staat van het spel lokaal bij en implementeert functies om de huidige staat van het spel grafisch af te beelden. Door middel van twee abstracte klassen: \emph{Object} en \emph{BoundingObject} generaliseren elke grafische modellering. De klasse \emph{Object} kan door gebruik te maken van de functie \emph{draw}, welke door de subklasse ge\"implementeerd moet worden, het object transleren en roteren naar een specifieke plaats in wereldco\"ordinaten. Ook heeft de klasse verwijzingen naar zogenaamde \emph{children}. Deze \emph{children} zijn ook weer van de klasse \emph{Object}. De plaats en rotatie die een \emph{child} heeft is relatief naar de plaats en rotatie van het hoofdobject. Door translatie en rotatie plaats het  hoofdobject deze ook op de juiste plek. \emph{BoundingObject} is een abstracte subklasse van \emph{Object}, deze kan bepalen of een bepaalde lijn, gedefinieerd door een punt en een richting, de \emph{BoundingBox} van het object snijdt. De \emph{BoundingBox} is een balk waarin het object compleet bevat moet zijn.
    
    De klasse 
    
    \subsection{Gebruikersomgeving}
    
    \subsubsection{Tijdens de opzet van het spel}
    
    \subsubsection{Tijdens het spel}
    
    \section{Onderlinge samenhang}
    
    \subsection{Gebruikersomgeving en protocol module tijdens de opzet van het spel}
    
    \subsection{Model van de sc\'ene en de gebruikersomgeving tijdens het spel}
    
    \section{Beschrijving klassen}
    
    \subsection{Model van de sc\`ene}
    \end{document}
