    \section{Beschrijving alternatieven}
    \label{app:alternatieven}
    Hier beschrijven we de door ons bekeken alternatieve ontwerpen voor een spel. Tijdens het brainstormen zijn we eerst de mogelijkheden gaan afbakenen door te kijken naar een aantal genres. Hierbij hebben we vooral gekeken naar persoonlijke voorkeuren. Verder hebben we natuurlijk rekening gehouden met het feit dat het spel een `voedsel' aspect moet hebben en dat het voor meerdere spelers moet zijn. Dit gaf de volgende genres:
    \begin{enumerate}
    \item[i] Strategie.
    \item[ii] Racen.
    \item[iii] First-Person Shooter.
    \item[iv] Levenssimulatie.
    \end{enumerate}
    We zullen nu elk genre dieper uitdiepen door de verschillende mogelijkheden verder in te vullen.

    \subsection{Strategie}
    We dachten bij strategie tijdens het brainstormen meteen aan het winnen van terrein als een vorm van `voedsel'. Het doel is dan om zoveel mogelijk terrein te winnen. Terrein wordt veroverd door een bepaalde tijd op een vakje te blijven staan. Verder dachten we al snel aan RTS, real-time strategie. Het idee is hier om `voedsel' te verzamelen om een leger te bouwen. Men wint dan door met dit leger de basis van de tegenstander te vernietigen.

    Via RTS kwamen we op het idee van \emph{Tower Defense}. Tower Defense wordt normaal gesproken met \'e\'en speler gespeeld. Vijandige monsters moeten dan worden tegen gehouden door torens te bouwen, die deze monsters voor jou aanvallen.

    \subsection{Racen}
    Bij een racespel dachten we aan twee alternatieven. De eerste optie was om het spelconcept van \emph{Mario Kart} ruwweg te volgen. Het doel hierbij is om zo snel mogelijk een vooraf vastgesteld aantal rondjes te rijden over een baan. Tijdens het rijden kunnen spelers power-ups verzamelen. De power-ups spelen hier dus de rol van het `voedsel'. Deze power-ups kunnen dan gebruikt worden om een voordeel te krijgen over andere spelers. Dit voordeel zou op zeer veel manieren kunnen worden gerealiseerd: zo zou een speler tijdelijk sneller kunnen gaan als gevolg van een power-up. Een andere mogelijkheid is de power-up af te kunnen schieten op andere spelers. Geraakte spelers kunnen dan worden vertraagd of tijdelijk stil komen te staan.

    Een tweede optie is dat spelers rond kunnen rijden in een grote stad. Het doel is dan om de auto's van alle andere spelers te vernietigen. Een auto van een tegenstander kan worden beschadigd door tegen de zijkant aan te rijden. Bovendien kan het spel verder worden uitgebreid, zodat de auto ook kan worden beschadigd door andere objecten.

    \subsection{First-Person Shooter}
    De First-Person Shooter is een alom bekend spelconcept. Een speler kan vrij over een gebied rondlopen. Meestal zal de speler een of ander wapen bij zich hebben. Met dit wapen kunnen andere spelers worden aangevallen. Het doel is dan om zoveel mogelijk medespelers uit te schakelen. Vaak is het zo dat een speler na een bepaalde tijd kan terugkeren in het spel, nadat die is uitgeschakeld.

    \subsection{Levenssimulatie}
    We dachten bij levenssimulatie aan een spel dat ge\"inspireerd is op het bekende spel \emph{Spore}. Ieder speler heeft in dit spel zijn eigen schepsel. Men kan zijn eigen schepsel laten groeien door andere schepsels, die niet noodzakelijk worden bestuurd door medespelers, op te eten. Het doel is dan om als enige over te blijven.

    \newpage
    \section{Motivering keuze}
    \label{app:motivering}
    In deze appendix motiveren we onze keuze uit de verschillende alternatieven in appendix \ref{app:alternatieven}. Tijdens het brainstormen was strategie meteen een van onze persoonlijke voorkeuren. Dit vonden wij allen een aantrekkelijk genre om te spelen. Het nadeel bij het veroveren van terrein is dat men al snel vastzit aan een grid. Dit is echter slechts een zeer kleine beperking. E\'en van onze verdere uitwerkingen van het genre strategie was RTS. Echter, RTS wordt al snel zeer complex om te maken, wat een groot nadeel is.

    Dit is ook de voornaamste reden dat we het simpelere Tower Defense hebben bekeken. Tower Defense is in onze mening relatief eenvoudig, maar toch aantrekkelijk om te spelen. De vraag was toen hoe Tower Defense kon worden uitgebreid tot een spel voor meerdere spelers. Het idee, dat hieruit kwam, is de basis voor het uiteindelijke spel, waarvan een korte samenvatting is uitgewerkt in het volgende hoofdstuk.

    Een racespel was ons voornaamste alternatief voor strategie. Ook dit vonden wij allen een aantrekkelijk genre om te spelen. We hadden hier, zoals al eerder besproken, ruwweg twee idee\"en: bij de eerste optie was het doel om zo snel mogelijk een vooraf bepaald aantal rondjes te rijden, bij de tweede optie was het doel om de auto's van alle andere spelers te vernietigen.

    Bij de eerste optie is er een mogelijk probleem dat power-ups, die in onze mening vitaal zijn om het spel leuk te maken, niet makkelijk kunnen worden ge\"implementeerd. Er zijn drie potenti\"ele problemen bij de tweede optie: zo is het niet duidelijk wat het `voedsel' aspect hier is. Bovendien is het zeker niet eenvoudig om te bepalen wie tegen wie aanrijdt. Als laatste is het visueel weergeven van de schade aan een auto lastig.

    Het spelconcept van First-Person Shooter hebben we niet uitgebreid bekeken tijdens het brainstormen. Ons voornaamste bezwaar was dat het zeer gecompliceerd is om te bepalen of iemand is geraakt door een kogel. Aangezien dit een zeer complex spelconcept is, willen wij dit zo simpel mogelijk houden. Daarom hebben we besloten dat spelers elkaar beschieten met lasers, die een snelheid van `oneindig' hebben. Zo kan dus direct bij het schot worden bepaald of het raak dan wel niet raak is. Dit vermijdt het probleem van bewegende objecten in de scene.

    Een groot voordeel is dat First-Person Shooters uit zichzelf al voor meerdere spelers zijn. Daarom is een deel van het First-Person Shooter concept ook opgenomen in het uiteindelijke spelconcept. Op deze manier kon het Tower Defense concept verder worden aangevuld tot een spel voor meerdere spelers: spelers kunnen elkaar aanvallen in het spel door elkaar te beschieten.

    Bij het idee van levenssimulatie werden wij ge\"inspireerd door het welbekende Spore, wat door een aantal van ons met enthousiasme is gespeeld. Er is ook een zeer duidelijk `voedsel' aspect aanwezig. Het was ons echter niet duidelijk hoe dit spel op een aantrekkelijke doch eenvoudige manier kon worden uitgebreid naar een spel voor meerdere spelers.

    \subsection{Het spel}
    Het uiteindelijke spel combineert een aantal van de eerdere idee\"en, met name Tower Defense en First-Person Shooter. We geven hier slechts een korte samenvatting van het spel. Een uitgebreide beschrijving van het spel staat in de spelspecificatie. We verdelen de spelers in twee teams. Spelers kunnen over het terrein rondlopen. Terwijl spelers rondlopen, kunnen ze torens bouwen met goud. Er zijn twee type torens: het eerste type toren kan op een mijn worden gebouwd en verzamelt extra goud. Het tweede type toren kan spelers aanvallen. Spelers kunnen andere spelers en torens aanvallen, dit is meteen ook de tweede manier om goud te verdienen. Als spelers doodgaan, komen ze een tijd later weer bij een speciaal gebouw terecht: het hoofdgebouw. Elk team heeft ook een hoofdgebouw, het doel van het spel is om het hoofdgebouw van het andere team te vernietigen.

    We hebben dus uiteindelijk gekozen voor een combinatie van de eerder genoemde spelconcepten. In onze mening is het daarmee een potentieel aantrekkelijk en innovatief spel. Het innovatieve element aan het spel is dat zowel elementen uit RTS als First-Person Shooter worden gecombineerd. Het grootste probleem is echter daarmee meteen de haalbaarheid van het spel. Elk spel moet al rekening houden met het netwerk aspect en bovendien met het gedistribueerde aspect, aangezien er geen server mag zijn tijdens het spel.  