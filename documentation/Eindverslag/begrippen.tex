\section{Begrippenlijst}
\label{app:begrippen}
De volgende tabel bevat een lijst met alle begrippen in het verslag:
    \begin{table}[H]
        \small
        \centering
        \begin{tabular}{| l | p{12cm} |}
        \hline
        Begrip & Betekenis \\ \hline
        Injectief & We noemen een functie $f: A \rightarrow B$ injectief als $f(a_1) = f(a_2) \Rightarrow a_1 = a_2$ voor
            alle $a_1, a_2 \in A$. \\ \hline
        Permutatie & Een permutatie is een bijectieve functie $\sigma: \{1, \ldots, n\} \rightarrow \{1, \ldots, n\}$
            voor een zeker natuurlijk getal $n$. \\ \hline
        Totaal geordend domein & We noemen $(A, \leq)$ een geordend domein als $\leq$ een reflexieve, transitieve en
            anti-symmetrische relatie is op $A^2$. Deze ordening noemen we totaal als voor alle $a_1, a_2 \in A$ geldt dat $a_1 \leq a_2$
            of $a_2 \leq a_1$. \\ \hline
        Bugs & Een fout in een programma. \\ \hline
        Mutual exclusion & Met mutual exclusion bedoelen we het probleem in de informatica waar twee processen nooit tegelijkertijd
            een kritieke sectie mogen uitvoeren. \\ \hline
        Token ring & Dit is een speciale graaf in de vorm van een ring, die alle processen bevat. Een token wordt dan doorgegeven
            door de processen in deze ring. Dit token representeert bij ons het recht om reliable messages te versturen. \\ \hline
        Reliable messages & Een speciaal soort bericht waarvoor mutual exclusion vereist is. \\ \hline
        Unreliable messages & Een bericht waarvoor geen mutual exclusion vereist is. \\ \hline
        Collision & Tijdens het schieten moet er worden bepaald of een object is geraakt. We zeggen dat we dan op zoek zijn
            naar een collision. \\ \hline
        Bounding box & Een balk om een object heen. Dit wordt gebruikt bij het rekenen aan collisions. \\ \hline
        Null & Een speciale waarde voor een variabele die aangeeft dat de variabele nog niet ge\"intialiseerd is. \\ \hline
        Bottom-up & Een methode voor het ontwerpen van complexe systemen. Hierbij worden eerst de modules ontworpen, die geen enkele andere
            module nodig hebben. Op deze basis bouwt men dan weer nieuwe modulen tot het uiteindelijke product is bereikt. \\ \hline
        Top-down & Het tegenovergestelde van Bottom-up.\\ \hline
        Operating system & Het besturingssysteem van een computer. \\ \hline
        Eerste persoon perspectief & Dit verwijst naar de plaatsing van de camera in het spel. Bij een eerste persoon perspectief
            bekijkt de speler de wereld vanuit zijn robot. De speler ziet dus de wereld zoals zijn robot die zou zien. \\ \hline
        Derde persoon perspectief & Bij een derde persoon perspectief bekijkt een speler de wereld vanaf een punt vlak achter
            zijn robot. \\ \hline
        Thread & Een thread is deel van een proces. Het is de eenheid van rekenen. \\ \hline
        Call-back & Een functie wordt meegegeven als argument voor een andere functie. \\ \hline
        Server & Een proces dat wacht op binnenkomende verbindingen om andere processen van dienst te zijn. \\ \hline
        Volledige graaf & Een graaf waarbij er een kant is tussen elke twee knopen. \\ \hline
        Fairness & Met fairness bedoelen we zekere eisen aan een systeem. De eis is dan dat toegang tot (een onderdeel van) het systeem
            op eerlijke wijze wordt verdeeld onder verschillende processen. \\ \hline
        Acknowledgement & Wij versturen een acknowledgement als reactie op reliable messages om de ontvangst van dit bericht te bevestigen. \\ \hline
        Frame rate & Het aantal getoonde beelden van een scherm gemeten in Hertz. \\ \hline
        \end{tabular}
        \caption{Begrippenlijst met betekenis}
        \label{tab:planning}
    \end{table} 