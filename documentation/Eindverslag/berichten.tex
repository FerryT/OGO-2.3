    \section{De berichten}
    \label{app:berichten}
    We zullen nu alle gebruikte berichten afgaan met een korte toelichting. De berichten zullen natuurlijk voldoen aan het berichtsformaat zoals gespecificeerd in appendix \ref{app:berichtsformaat}. We zullen vooral kijken naar het nut voor de ontvanger. Om de berichten te beschrijven gebruiken we een aantal conventies. We zullen voor de leesbaarheid komma's plaatsen tussen de verschillende argumenten. Deze zijn dus niet deel van het bericht. Ook zullen we de commando's laten eindigen met een punt, dit moet dan worden gelezen als \textsc{cr lf}.

    \subsection{Server detectie}
    Het \emph{Broadcast} bericht wordt gebruikt voor het detecteren van servers tijdens de initialisatie. Dit bericht word periodiek door servers verstuurd met behulp van broadcast over \udp. We hebben ervoor gekozen om dit bericht elke vijf seconden te sturen. De wachttijd voor de gebruikers is dan nog steeds verwaarloosbaar, terwijl het netwerk nauwelijks wordt belast.

    Spelers moeten continu voor deze Broadcast berichten scannen. De gevonden servers worden dan in een lijst gezet. De speler kan dan uit deze lijst van servers kiezen. Hierdoor kunnen processen in hetzelfde subnet een server vinden. We geven ook de mogelijkheid om direct het ip van de server op te geven. Dit heeft het grote voordeel dat het mogelijk wordt om ook te spelen met spelers buiten het lokale netwerk. Merk op dat de server natuurlijk alleen wordt gebruikt voor het opzetten van het spel.

    Het bericht bevat de versie van het spel en het huidige aantal spelers. Bovendien bevat het de naam van het spel. Het bericht heeft de volgende syntax:
    \bericht{GOTO, version, numPlayers :gameName.}

    \subsection{De lobby}
    Het \emph{Name} bericht wordt gebruikt na de server detectie. Als een speler de lobby van een bepaald spel wil binnengaan, wordt dit gedaan door een Name bericht te sturen. In dit bericht aan de server stuurt de speler zijn gewenste naam mee:
    \bericht{NAME :playername.}
    Indien de speler wordt toegelaten, zal de server een \emph{Hello} bericht terugsturen als reactie op het Name bericht (en wordt dus alleen gestuurd aan de speler die het Name bericht heeft gestuurd aan de server). Hierin staat het toegewezen ID van de speler, de versie, het aantal spelers en de naam van het spel. Dit geeft het volgende bericht:
    \bericht{HELLO, pid, version, numPlayers :gameName.}
    Na het versturen van het Hello bericht, zal de server ook meteen een aantal \emph{Player} berichten sturen. Dit bericht wordt aan de nieuwe speler gestuurd om het team en status van elke medespeler kenbaar te maken. De status kan B zijn voor niet ready, R zijn voor ready of H zijn voor host. Alle spelers zitten standaard in hetzelfde team. Bovendien hebben alle spelers standaard de status niet ready. Het spel kan pas beginnen als alle spelers ready zijn. Dan kan de speler, die de server draait, op start drukken om het spel te starten. Het bericht heeft de volgende syntax:
    \bericht{PLAYER, pid, tid, state :playerName.}
    Na het versturen van de Hello en Player bericht, zal de server aan alle spelers een \emph{Join} bericht sturen. Hierdoor weten alle spelers dat er een nieuwe medespeler is bijgekomen. Dit bericht wordt ook aan de nieuwe speler zelf gestuurd. Hierdoor kan de nieuwe speler concluderen dat de Player berichten zijn afgelopen. Dit bericht bevat de ID van de speler en de naam van de speler:
    \bericht{JOIN, pid :playerName.}
    Het \emph{Part} bericht wordt gestuurd als de verbinding van een speler met de server, al dan niet vrijwillig, wordt verbroken. Dit bericht wordt aan iedereen gestuurd om mede te delen dat een van de medespelers is vertrokken. We sturen alleen de ID van de vertrokken speler mee:
    \bericht{PART, pid.}
    Met een \emph{Team request} bericht kan de speler een verzoek naar de server sturen om van team te wisselen. Dit bericht bevat dan het nieuwe gewenste team van de speler:
    \bericht{TEAM, tid.}
    Als een speler van team is gewisseld, wordt dit kenbaar gemaakt door de server met het \emph{Team} bericht. De server stuurt dan aan de andere spelers het volgende bericht om dit kenbaar te maken:
    \bericht{TEAM, pid, tid.}
    Verder zijn er nog \emph{Ready request} en \emph{Busy request} berichten. Een speler kan de ready en busy berichten gebruiken om zijn status te veranderen. Ready betekent dat de status ready moet worden, terwijl Busy betekent dat de status niet ready moet worden. Impliciet wordt er een afbeelding bijgehouden tussen het ip-adres, de speler ID en de naam van de speler. Dit geeft de volgende twee berichten:
	\bericht{READY.}
    \bericht{BUSY.}
    Nadat een speler van team is gewisseld, maakt de server dit kenbaar aan de andere spelers. Dit wordt gedaan door de ID van de speler mee te sturen:
    \bericht{READY, pid.}
    \bericht{BUSY, pid.}
    Een speler kan chatten met de andere spelers. Dit wordt gedaan door een \emph{Chat request} bericht te sturen naar de server. Dit bericht bevat dan het chatbericht:
    \bericht{CHAT :msg.}
    Zodra de server een Chat request bericht ontvangt, stuurt de server een \emph{Chat} bericht naar alle andere spelers. Dit bericht bevat de ID van de speler en het chatbericht:
    \bericht{CHAT, pid :msg.}
    Merk op dat er soms twee commando's zijn met dezelfde naam. Deze zijn echter zeer eenvoudig van elkaar te onderscheiden. In dit geval is er namelijk voor gezorgd dat \'e\'en commando alleen door de clients wordt verzonden en het andere commando alleen door de server. Hierdoor ontvangt elke speler nooit meer dan \'e\'en van de twee type commando's.

\subsection{Berichten om het spel op te zetten}
Om het spel op te kunnen zetten hebben we ook speciale berichten nodig. Met behulp van deze berichten proberen we een volledige graaf en een token ring op te bouwen. Met een \emph{Good-day} bericht maakt de speler contact met de server. Zo kan de speler toegevoegd worden aan de graaf. \textsc{tid} identificeert het team en \textsc{name} geeft de speler een naam.
\bericht{GOODDAY, tid :name.}

Nadat de server een \emph{Good-day} bericht heeft ontvangen, stuurt de server een \emph{Welcome} bericht als antwoord. \textsc{pid} wijst de speler een uniek ID toe. Version is een parameter om af te dwingen dat alle spelers de zelfde versie hebben. \textsc{iplist} is een string van alle ip-adressen in de standaard \textsc{cidr} notatie (dat wil zeggen dat een adres gerepresenteerd wordt in de vorm: \textsc{xxx.xxx.xxx.xxx}). Tussen elk ip-adres wordt het karakter \emph{e} gebruikt om ip-adressen te onderscheiden.
\bericht{WELCOME, pid , version :iplist.}

Het \emph{Sup} bericht wordt gestuurd door een speler zodra hij een verbinding met een andere speler aanmaakt. \textsc{pid} geeft de speler ID van de speler die het bericht verstuurt aan en \textsc{version} zorgt er weer voor dat de versies overeenkomen. \textsc{name} geeft de naam van de speler aan.
\bericht{SUP, pid, version :name.}

Het \emph{Gladtomeetyou} bericht is een bevestiging van het \emph{Sup} bericht. Hiermee geeft de speler, welke het \emph{Sup} bericht ontving, aan wat zijn naam en speler ID is. \textsc{pid} is dan weer de speler ID en name de naam van de speler.
\bericht{GLADTOMEETYOU, pid :name}

\subsection{Berichten tijdens het spel}
De volgende berichten worden gebruikt voor communicatie tijdens het spel. Deze berichten worden dan naar alle andere spelers gestuurd. Sommige berichten zijn hierbij aangegeven met een R. Deze berichten krijgen een speciale behandeling, zoals wordt besproken in sectie \ref{sec:tijdensspel}. Als eerste is er een zogenaamd \emph{GameChat} bericht. Dit bericht vervult een vergelijkbare taak als het Chat bericht in de lobby. Het GameChat bericht heeft de volgende syntax:
\bericht{GAMECHAT, message.}

Een speler zal periodiek zijn huidige positie doorsturen. Bovendien wordt hierbij de huidige richting meegegeven. Hiervoor wordt het \emph{Move} bericht verstuurd:
\bericht{MOVE, x, y, z, vx, vy, vz.}

Als een speler heeft geschoten, dan wordt dit naar alle andere spelers gestuurd. Hierbij wordt de oorsprong van het schot en de richting van het schot meegestuurd. Merk op dat deze overeenkomen met respectievelijk de positie van de speler en de kijkrichting. De bedoeling van dit bericht is enkel om de bijbehorende animatie weer te geven: eventuele schade aan het harnas wordt apart afgehandeld.
\bericht{FIRE, x, y, z, vx, vy, vz.}

De spelers houden lokaal de totale hoeveelheid goud van het team bij. Spelers zullen deze waarde periodiek versturen. Als deze waarden tussen verschillende spelers van hetzelfde team niet overeenkomen, zullen we deze waarden middelen. Het bericht \emph{Team} verstuurt de ID van het team en de hoeveelheid goud:
\bericht{TEAM, tid, teamGold.}

Verder is er nog een \emph{Respawn} bericht. Dit bericht wordt verstuurd nadat een speler is dood gegaan. In dit bericht wordt de nieuwe locatie meegestuurd, dit zal in principe het commandocentrum van het bijbehorende team zijn:
\bericht{R RESPAWN, x, y, z.}

Na een schot zal een speler berekenen of een speler van het andere team is geraakt. Als dit het geval is, dan zal de geraakte speler en de hoeveelheid schade worden verstuurd. Merk op dat dit lokaal wordt berekend. Het hoeft niet noodzakelijk te zijn dat het schot van de speler zelf komt, het kan ook komen van een gebouw in het bezit van een speler. De geraakte speler en de hoeveelheid schade wordt doorgegeven in het bericht \emph{Hit}:
\bericht{R HIT, pid, damage.}

Als een speler is doodgegaan, zal hij dit kenbaar maken aan alle andere spelers. Dit wordt gedaan door het \emph{Died} bericht, waarin ook de dader is opgenomen. Op zich is de informatie over de dader irrelevant voor het spel, maar dit kan wel worden gebruikt om het spel verder uit te breiden.
\bericht{R DIED, pid.}

Een speler zal een voorwerp achterlaten bij het doodgaan. In het huidige ontwerp zal dit altijd een muntje zijn. Dit muntje krijgt meteen ook een unieke ID en een zekere waarde. Verder zal nog de lokatie van het muntje worden meegestuurd. Als laatste zal nog het type voorwerp worden meegestuurd, dit argument kan worden gebruikt voor verdere uitbreiding. Samen geeft dit het \emph{Drop} bericht:
\bericht{R DROP, id, x, y, z, value, type.}

Als een speler een voorwerp opraapt, zal er een \emph{Take} bericht worden verstuurd. Hierbij zal natuurlijk ook de ID van het opgeraapte voorwerp worden meegestuurd.
\bericht{R TAKE, id.}

Een speler kan natuurlijk ook een gebouw bouwen. Het spel heeft de lokale verantwoordelijkheid om te controleren dat de bouwplaats toegestaan is. In het \emph{Build} bericht wordt naast de lokatie ook nog een uniek ID en het type van het gebouw meegestuurd:
\bericht{R BUILD, id, x, y, z, type.}

Als een speler een gebouw heeft geraakt met een schot, dan zal de ID van het gebouw worden verstuurd en de hoeveelheid schade. Dit wordt gedaan in het \emph{Attack} bericht:
\bericht{R ATTACK, id, damage.}

Op een gegeven moment kan het ook gebeuren dat het gebouw is vernietigd. De eigenaar van het gebouw heeft de verantwoordelijkheid om dit te controleren. Als het gebouw is vernietigd, dan wordt de ID van het gebouw verstuurd. Bovendien zal de dader hierbij worden verstuurd net zoals in het geval dat een speler is doodgegaan. Hiertoe wordt het \emph{Destroy} bericht gebruikt:
\bericht{R DESTROY, id, pid.}

Als laatste is er nog het \emph{End} bericht. Dit bericht wordt verstuurd zodra een team heeft gewonnen: oftewel als het commandocentrum van het andere team is vernietigd. In dit bericht wordt het winnende team meegestuurd:
\bericht{R END, tid.} 