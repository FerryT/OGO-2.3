Informatici in de hedendaagse wereld komen vaak in aanraking met het ontwerpen van complexe programma's. Complexe programma's hebben vaak ook een netwerk aspect en een grafische aspect. Zelfs bij technische programma's, die vaak minder grafisch intensief zijn, als matlab worden al deze aspecten verenigd. Dit drukt de noodzaak uit dat elke informaticus basiskennis heeft van computergrafiek, computernetwerken en het ontwerpen van complexe programma's. Al deze aspecten worden verenigd in dit project: \textsc{ogo} 2.3 - Nethunt. 

Voor dit project werd ons gevraagd om een interactief, gedistribueerd 3D-spel te ontwikkelen. Dit houdt in dat elk speler lokaal dezelfde spelsituatie heeft, en deze op het scherm van de speler worden afgebeeld. Natuurlijk kan de afbeelding op het scherm verschillen per speler. Een andere eis was dat er voedsel aanwezig moet zijn. E\'en van de randvoorwaarden was dat elke speler ... INSERT MORE

Dit document bevat de documentatie voor dit project. We zullen beginnen door een strakke planning te geven in de vorm van een werkplan. Daarna gaan we verder met een specificatie van het spel. Hierbij beschrijven we ook de alternatieven en de motivering voor onze keuze. We zijn dan klaar om een ontwerp van het spel te geven, inclusief het communicatieprotocol. De impliciete aannamen in het communicatieprotocol zullen we proberen te identificeren. We sluiten af met een validatie van deze aannamen en een motivering voor de implementatie, gevolgd door een conclusie en evaluatie.