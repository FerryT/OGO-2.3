    \section{Het ontwerp}
    % Inclusief alternatieven, motivering van keuzes, aannames en communicatieprotocol
    Nu zijn we klaar om het ontwerp van ons spel \textsc{GOTO} te bespreken. We zullen beginnen door de verschillende componenten te analyseren. Het programma is hierbij verdeeld in drie zo onafhankelijk mogelijke componenten. Wij onderscheiden de volgende componenten:
    \begin{itemize}
    \item Een component die het communiceren naar andere spelers mogelijk maakt;
	\item Het model van de sc\`ene. Denk hierbij aan de staat, locatie en grafische modellen van gebouwen, spelers en het terrein;
	\item De gebruikersomgeving die de interactie van de speler met het spel mogelijk maakt en ook het grafische model weergeeft tijdens het spel.
    \end{itemize}

    Als uitgangspunt ontwerpen wij de verschillende componenten zodanig dat elke component zo onafhankelijk mogelijk ontwikkeld kan worden.
    	
    \subsection{Protocol module}
    De protocol module is het onderdeel die het mogelijk maakt om:
	\begin{itemize}
    \item Een nieuw spel op te starten. Het spel komt dan in de zogenaamde opstart fase. Andere spelers kunnen dan aan dit spel meedoen, ze komen dan in de lobby van dat spel;
	\item De lijst van alle spellen, die nog in de opstart fase zijn, in het subnet van de computer weer te geven;
	\item Mee te doen met een bestaand spel dat nog in de opstart fase zit;
	\item Het spel te beginnen;
	\item Mutaties van de staat van het spel te ontvangen en te versturen.
	\end{itemize}
    De manier, waarop deze communicatie verloopt, is beschreven vanaf \protoref. De gebruikersomgeving kan de protocol module een opdracht geven. Deze    opdracht zal dan asynchroon worden uitgevoerd door de protocol module. Dit betekent dat het protocol uit zichzelf berichten kan ontvangen en verwerken. Door middel van threads wordt er voor gezorgd dat het protocol de taken onafhankelijk kan uitvoeren.
	
    \subsection{Het model van de sc\`ene}
    Het model van de sc\`ene vervult twee rollen. Het houdt de huidige staat van het spel bij en zorgt ervoor dat de huidige staat van het spel afgebeeld kan worden op het scherm. Het model is dus in principe een passief element, het zal niet uit zichzelf acties uitvoeren (uitzonderingen daargelaten). Een diepgaande beschrijving van het model kan gevonden worden in appendix \ref{app:klassenbeschrijving}.

    \subsection{Gebruikersomgeving}
    De gebruikersomgeving moet zowel het grafische model van het spel afbeelden op het scherm als de interactie van de speler via het toetsenbord en muis afhandelen. Ook moet de gebruikersomgeving het mogelijk maken voor de gebruiker om een spel op te starten of mee te doen aan een spel in de opstart fase.

    \subsubsection{De opzet van het spel}
    Tijdens de opzet van het spel zal de gebruikersomgeving een venster weer moeten geven met alle spellen in het subnet van de computer. De gebruiker kan dan kiezen om mee te doen aan een van deze spellen of zijn eigen spel te openen. Als de gebruiker zelf een spel opent, kunnen andere gebruikers in hetzelfde subnet ook meedoen aan dit spel. Deze spelers kunnen dan met elkaar communiceren in een lobby. Hierin kunnen zij ook aangeven of zij klaar zijn om het spel te starten. Zodra alle gebruikers hebben aangegeven dat zij klaar zijn, kan het spel worden gestart. De keuze om het spel te starten is dan aan de gebruiker, die ook het spel heeft geopend.

    \subsubsection{Tijdens het spel}
    De gebruikersomgeving zal tijdens het spel via \textsc{openGL} de lokale staat van het spel afbeelden op het scherm. Dit wordt gedaan door de \emph{render} methode van het object \emph{World} aan te roepen, die is beschreven in sectie \ref{sec:model}. De gebruikersomgeving zal daarna alleen nog menu's, mogelijke kaarten en informatie als de sterkte van het harnas en de hoeveelheid geld op het scherm moeten aangeven (zoals beschreven in het document Informele Beschrijving). Door de 2D-modus van openGL te gebruiken kan dit gemakkelijk getekend worden boven op de grafische representatie van de wereld.

    De gebruikersomgeving zal ook het model aanpassen aan de hand van de interactie van de gebruiker met de omgeving. Om dit voor elkaar te krijgen slaat
    de gebruikersomgeving alle muisbewegingen, klikken en toetsenbord aanslagen op en past het model op de bijbehorende manier aan. Een verdere uitleg van de interactie tussen de gebruikersomgeving en het model is gegeven in sectie \ref{sec:interactie}.

    \section{Onderlinge samenhang}
    \label{sec:protocol}
    De drie onderdelen, die wij hierboven beschreven hebben, moeten natuurlijk allemaal met elkaar kunnen communiceren. Deze communicatie zullen we nu beschrijven.

    \subsection{Gebruikersomgeving en protocol module tijdens de opzet van het spel}
    De gebruikersomgeving en de protocol module communiceren tijdens de opzet van het spel compleet asymmetrisch. Dat wil zeggen dat het protocol op elk moment een bericht kan sturen naar de gebruikersomgeving en ook andersom. Om dit mogelijk te maken, heeft de protocol module verschillende functies ge\"implementeerd, die aangeroepen kunnen worden door de gebruikersomgeving. Deze kunnen bijvoorbeeld gebruikt worden om een nieuw spel aan te maken of mee te doen aan een bestaand spel. Op deze manier kan de gebruikersomgeving acties van de gebruiker omzetten in een actie van het protocol: de protocol module handelt dit dan af.

    De protocol module luistert actief naar binnenkomende berichten. Aangezien het protocol real-time verplichtingen heeft, zal de module gebruik maken van meerdere threads. Zo fungeert de module los van de gebruikersomgeving. Als er een bijzondere gebeurtenis heeft plaatsgevonden, kan het protocol dit doorgeven aan de gebruikersomgeving door middel van zogenaamde \emph{call-backs}. Dit zijn methoden die de gebruikersomgeving implementeert, zodat de protocol module informatie kan doorgeven aan de gebruikersomgeving. De gebruikersomgeving zal deze informatie zodanig moeten afhandelen dat het afgebeeld kan worden op het scherm.

    Niet alle communicatie van de protocol module en de gebruikersomgeving verloopt via call-backs en methoden, sommige informatie zal de protocol module namelijk niet als een gebeurtenis doorgeven. Deze informatie wordt louter opgeslagen in de protocol module en kan de gebruikersomgeving opvragen. Denk hierbij aan de lijst van spellen op het subnet, spelers in het spel, enzovoorts. Een voorbeeld van interactie van de gebruikersomgeving en protocol module is gegeven in een \textsc{msc} in appendix \ref{app:MSCLobbyCon}.

    \subsection{Model van de sc\`ene en de gebruikersomgeving tijdens het spel}
    \label{sec:interactie}
    Het model van de sc\`ene zal niet actief communiceren met de gebruikersomgeving tijdens het spel. Dat wil zeggen dat er geen asymmetrische communicatie plaatsvindt via call-backs. De gebruikersomgeving zal aan de hand van de invoer van de gebruiker het model aanpassen. De gebruikersomgeving zal ook de grafische representatie van de staat van het spel uit het model halen. Het model maakt dit mogelijk door de beschreven \emph{render} methode in de interface \emph{object}. Het model zorgt voor enkele globale methoden die in \'e\'en keer een actie van de gebruiker kan verwerken in het model. Hierdoor hoeft de gebruikersomgeving alleen een paar globale methoden aan te roepen. Hiermee voorkomen wij dat de gebruikersomgeving veel gebruik moet maken van de structuur van het model. Zo wordt de gebruikersomgeving minder afhankelijk van de implementatie van het model.

    We hebben nu de verschillende componenten en hun onderlinge communicatie uitgediept. Een volledige beschrijving van de klassen wordt gegeven in appendix \ref{app:klassenbeschrijving}. Er is ook een bijbehorend klassediagram gemaakt. Het klassediagram is te vinden in appendix \ref{app:klassendiagram}. We zullen nu verder gaan met de beschrijving van het protocol.

    \section{De opbouw van de graaf}
    Het protocol zal een volledige graaf proberen op te bouwen. Het berichtsformaat wordt formeel gedefinieerd in appendix \ref{app:berichtsformaat}. De bijbehorende berichten zijn beschreven in appendix \ref{app:berichten}. Dit wordt gedaan tijdens de lobby fase van het spel. We weten dat er tijdens de lobby fase een server is. Elk proces probeert nu een \tcp-connectie te maken met deze server. Zodra deze \tcp-connectie is geslaagd, zal het proces een lijst teruggestuurd krijgen. Deze lijst bevat alle adressen van processen, waarmee de server nu is verbonden. Verbindingen uit de lobby fase tellen hierbij dus niet mee. Het proces is dan zelf verantwoordelijk om een \tcp-connectie aan te gaan met elk proces in deze lijst.

    Dit betekent dat op elk willekeurig moment een aanvraag voor een \tcp-connectie kan binnenkomen bij processen, die al een \tcp-connectie met de server hebben. Kortom, elk proces zal regelmatig moeten controleren op aanvragen voor een \tcp-connectie. Als de volledige graaf is opgebouwd, staan alle processen op gelijke voet. Op dat moment kan het spel beginnen.

    \section{Het protocol tijdens het spel}
    \label{sec:tijdensspel}
    De staat van het spel wordt uniek vastgelegd door de waarde van alle variabelen, inclusief de toestand van alle \tcp-connecties. Het is echter ondoenlijk om alle waarden van de variabelen voortdurend te versturen over de \tcp-connecties. Daarom zal elke speler een replica van het spel hebben. Deze replica's moeten natuurlijk zodanig worden gesynchroniseerd dat spelers er niet achterkomen dat hun replica's op kleine details tijdelijk kunnen verschillen.

    \subsection{Reliable en unreliable variabelen}
    We maken daarom onderscheid tussen twee soorten variabelen, \emph{reliable} en \emph{unreliable}. Reliable variabelen moeten gelijk zijn voor alle replica's om inconsistentie te voorkomen. Unreliable variabelen mogen wel verschillen tussen de spelers. Deze kunnen dus worden `gesimuleerd' door spelers. Om de reliable variabelen consistent te houden hebben we zogenaamde \emph{mutual exclusion} nodig. Een goed voorbeeld hiervan is het oprapen van een muntje.

    We pakken dit probleem aan door de notie van autoriteit te introduceren. Op elk moment heeft precies \'e\'en proces autoriteit. Als het proces autoriteit heeft, dan mag het proces reliable variabelen veranderen. Hiervoor wordt een bericht naar elk ander proces gestuurd. Het proces wacht dan totdat iedereen heeft teruggestuurd dat dit bericht is ontvangen. Als het proces alle wijzigingen van reliable variabelen heeft doorgegeven, krijgt een ander proces autoriteit.

    \subsection{De token ring}
    De volgende vraag is nu hoe we deze notie van autoriteit gaan verwerken in het protocol. Dit doen we door gebruik te maken van een \emph{token ring}. We bouwen tijdens de lobby fase (opstart fase) ook een ring op tussen alle processen. Dit doen we pas nadat de volledige graaf al is opgebouwd. Een mogelijke aanpak om dan een ring op te bouwen is om aan ieder proces een verschillend element uit een totaal geordend domein toe te kennen.

    We kunnen dit beschouwen als een injectieve functie $f$ van de verzameling processen $P$ naar een totaal geordende domein. Aangezien er sprake is van een totaal geordend domein, kunnen we deze waarden sorteren. Stel nu dat $P = \{ p_1, \ldots p_n\}$. Dit geeft ons een permutatie $\sigma$ zodanig dat:
    \[
    f(p_{\sigma(1)}) < f(p_{\sigma(2)}) < \ldots < f(p_{\sigma(n)})
    \]
    We verbinden nu $p_{\sigma(1)}$ met $p_{\sigma(2)}$, $p_{\sigma(2)}$ met $p_{\sigma(3)}$ enzovoorts. Bovendien verbinden we $p_{\sigma(n)}$ met $p_{\sigma(1)}$. We zijn er nu dus in geslaagd om de ring te construeren. Voor het totaal geordende domein kunnen we IP-adressen gebruiken, zodat uniciteit is gegarandeerd.

    Zodra een proces het token heeft, heeft het proces autoriteit. Het kan dan alle gewenste veranderingen van de reliable variabelen doorsturen over de volledige graaf. De andere processen zullen dan een \emph{acknowledgement} terugsturen dat dit bericht is ontvangen. Zodra alle acknowledgements van de andere processen zijn ontvangen, zal het proces de token zo snel mogelijk doorsturen naar het volgende proces in de ring. Dit zorgt ervoor dat elk proces binnen een eindige hoeveelheid tijd de token krijgt. Er is dus een zekere vorm van \emph{fairness}.

    \subsection{Simulatie}
    Sommige acties mogen ook worden uitgevoerd wanneer het desbetreffende proces niet de token heeft. Een speler mag bijvoorbeeld vrij rondlopen zonder dit te vragen aan alle andere spelers. Dit kan ervoor zorgen dat niet alle processen exact dezelfde staat hebben. Het is echter voldoende als alle processen naar dezelfde staat convergeren. We zullen dit verder toelichten aan de hand van een voorbeeld.

    Een proces zal de positie van alle andere spelers simuleren. Dit wordt gedaan met behulp van de locatie en de looprichting. De actuele locatie en looprichting van de desbetreffende speler zal periodiek over de volledige graaf worden verstuurd. Om eventuele problemen te voorkomen zou deze periode redelijk klein moeten zijn. In de tussentijd zal het proces de huidige locatie en looprichting van de speler benaderen door de speler simpelweg `verder' te laten lopen. Merk op dat op deze manier de speler ten alle tijden de baas blijft over zijn eigen locatie.

    \section{Aannamen}
    We hebben al eerder genoemd dat het \emph{Broadcast} bericht over \udp wordt gestuurd. Hier gebruiken we de broadcast functionaliteit van \udp voor. Het is dus noodzakelijk dat alle spelers zich op hetzelfde subnet bevinden. We weten dat \udp-berichten niet noodzakelijk aan hoeven te komen. We nemen echter aan dat een \udp-bericht uiteindelijk wordt ontvangen binnen een eindige hoeveelheid tijd, als de server periodiek \udp-berichten blijft versturen.

    Een speler kan dan een keuze maken tussen alle servers, waarvan een \emph{Broadcast} bericht is ontvangen. De speler gaat dan een \tcp-connectie aan met de uitgekozen server. We nemen aan dat deze \tcp-connectie succesvol wordt aangemaakt. Bovendien nemen we aan dat \tcp inderdaad voldoet aan de FIFO-eigenschap. Verder nemen we aan dat berichten over \tcp ook aankomen, eventueel nadat dit bericht een aantal keer opnieuw is verstuurd.

    Verder maken we nog een aantal simpele aannamen over de snelheid van het netwerk en de computers. De gebruikte berichten zijn allemaal zeer klein. We verwachten daarom dat er elke seconde zonder problemen 300 berichten over het netwerk kunnen worden verstuurd. Dit zou ruim voldoende moeten zijn om de replica's goede benaderingen van de werkelijke staat te laten zijn. Hiermee bedoelen we met een goede benadering dat het verschil met de werkelijke staat niet zou moeten opvallen voor de spelers.

    De helft van de RTT, oftewel de \emph{Round-Trip Time}, is een goede benadering voor hoe lang het duurt om een bericht te versturen. We gaan ervan uit dat de RTT kleiner is dan 200 ms. Deze aanname is zeker realistisch, aangezien de spelers in principe op hetzelfde subnet zitten. Vanwege deze aanname krijgen we ook een harde grens op de maximale afwijking van de replica's.
    \label{app:MSCLobbyCon} 