\section{Optioneel}
\label{app:optioneel}
Hier bespreken we onze plannen voor uitbreiding van het spel. De volgende idee\"en zouden wij graag extra implementeren. De idee\"en zijn gesorteerd op aflopende volgorde van belangrijkheid.

\begin{itemize}
  \item Extra goud, de gezamenlijke kas wordt periodiek met een vaste hoeveelheid verhoogd. Dit is dan additief met eventuele mijnen. Wij verwachten dat dit weinig tijd kost.
  \item Oorlogsmist, de robots hebben een beperkt zichtveld. Dit voorkomt dat spelers vanaf hun commandocentrum alle activiteiten van de tegenstanders kunnen zien. Wij denken dat dit weinig tijd kost.
  \item Plattegrond, zodat spelers een globaal overzicht krijgen van wat er op de kaart gebeurt. Dit kan eventueel ook weer met een oorlogsmist zijn, zodat de spelers alleen een gedeelte van het plattegrond kunnen zien, waar teamgenoten in de buurt staan. Wij vermoeden dat dit niet veel tijd kost.
  \item Muren, als een extra type gebouw. Onze bedoeling hierbij is dat deze muren relatief goedkoop zijn om te bouwen. Dit geeft een team meer opties om hun commandocentrum of delfplaatsen te beschermen. Bovendien geeft het de extra mogelijkheid om een veilige plek te cre\"eren, waarvandaan tegenstanders aangevallen kunnen worden. Wij voorzien dat dit een kleine hoeveelheid tijd kost.
  \item Ontwikkelingen, de spelers krijgen ontwikkelingspunten tijdens het spel. Deze kunnen worden verdiend door het uitschakelen van spelers, werkers, infanterie en het vernietigen van gebouwen. Werkers en infanterie worden uitgelegd in de volgende twee idee\"en. Hiermee kunnen de spelers bijvoorbeeld de eigenschappen van hun robot verbeteren, zoals de snelheid van lopen en hoe sterk het harnas is. Ook is het mogelijk om sterkere wapens te kopen. Wij denken dat dit veel tijd kost, aangezien een `winkel' gemaakt moet worden waar deze punten gespendeerd kunnen worden. Ook moeten de eigenschappen van een speler dynamisch gemaakt worden.
  \item Werkers, dit zijn computergestuurde robots. Zodra een speler daartoe opdracht geeft, komen werkers uit het commandocentrum van het bijbehorende team om een gebouw neer te zetten. Dit vervangt de oude mogelijkheid van spelers om te bouwen. Een gevolg van deze aanpassing is dat het langer duurt om een gebouw te bouwen als de afstand van de bouwplaats tot het commandocentrum groter is. Bovendien wordt het mogelijk om het bouwen van het andere team te vertragen door de werkers uit te schakelen. Dit kost heel veel tijd, aangezien de werkers gedistribueerd bestuurd worden. Hiervoor is dus een soort gedistribueerde kunstmatige intelligentie nodig. Dit is een uitdaging. Merk ook op, dat deze computergestuurde robots niet door de gebouwen horen te lopen, waardoor deze kunstmatige intelligentie totaal niet triviaal is.
  \item Infanterie, ook dit zijn computergestuurde robots. Deze kunnen door spelers worden gekocht, vanaf het commandocentrum lopen ze naar gebouwen van het andere team om deze gebouwen aan te vallen. Dit kost nog meer tijd dan de werkers, aangezien de infanterie niet tussen twee vaste punten zich moeten verplaatsen, maar ook gebouwen moeten kunnen aanvallen.
\end{itemize} 