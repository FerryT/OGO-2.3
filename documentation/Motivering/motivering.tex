\section{Motivering voor de implementatie}
Nu zullen we onze keuzes bij de implementatie toelichten. Zo hebben wij ervoor gekozen om C++ te gebruiken. Daar zijn twee belangrijke redenen voor. Ten eerste is C++ snel. Dit is natuurlijk een voordeel bij het maken van een spel. Hierdoor kunnen ook mensen met relatief trage systemen het spel draaien zonder dat dit ten koste gaat van de grafische aantrekkelijkheid.

Ten tweede was er nog het educatieve aspect. Een deel van de groep was namelijk nog niet bekend met C++ aan het begin van het project. Deze groepsleden leren hiermee een nieuwe programmeertaal naast Java. Het is altijd handig om kennis te hebben van verschillende programmeertalen.

Verder gebruiken we \emph{WxWidgets} bij de implementatie. De reden hiervoor is de grote diversiteit aan \emph{Operating Systems}: we gebruiken zowel Windows, Mac en Linux. Een groot voordeel van WxWidgets is dat het op al deze Operating Systems werkt. Ook gebruiken we nog \emph{Freeglut} en \emph{OpenGL}. Freeglut is een open source alternatief voor GLUT. Het voordeel van Freeglut is dat een aantal bugs in GLUT zijn verholpen in Freeglut.

Als laatste noemen we nog \emph{Autodesk 3ds Max}. Dit programma is gebruikt voor het modelleren van de sc\`ene. Hierin zijn alle textures en modellen gemaakt. Daarbij hebben we nog \emph{lib3ds} gebruikt om de textures en modellen in te laden. 