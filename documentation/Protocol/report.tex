\documentclass[a4paper,11pt]{article}
\usepackage{graphicx,listings,float,geometry, amsmath}
%\usepackage[firstpage]{draftwatermark}
%\SetWatermarkLightness{0.5}
%\SetWatermarkScale{4}
\setcounter{tocdepth}{2}
\usepackage[dutch]{babel}

\geometry{
	includeheadfoot,
	margin=2.54cm
}

\begin{document}
	\begin{titlepage}
	\begin{center}
		
		{\Huge Informele Specificatie \\[0.5cm]OGO 2.3 - Multiplayer Game}\\[0.5cm]
		\rule{\linewidth}{0.5mm}\\[0.5cm]
				\bigskip
		\huge \textit{``Grudge of the Oblivious''}
		
		{\Large
		Luca van Ballegooijen, Tim van Dalen, \\
		Carl van Dueren den Hollander, Peter Koymans,\\
		Kay Lukas en Ferry Timmers\\[1cm]
		}
		
		{\large
		OGO 2.3\\
		Groep 3 \\[1cm]
		Faculteit Wiskunde en Informatica\\
		Technische Universiteit Eindhoven\\[1cm]
		}
		
		\begin{abstract}

    In dit document zullen we de kritieke punten bij dit project identificeren. Vervolgens bekijken we de taken die bij dit project een rol spelen. Tenslotte zullen we dit gebruiken om een werkplan voor het project op te stellen.
\end{abstract}


		\vfill

		{\large \today}
	\end{center}
\end{titlepage}

	
	\tableofcontents
	\newpage

	\section{Introductie}
	Informatici in de hedendaagse wereld komen vaak in aanraking met het ontwerpen van complexe programma's. Complexe programma's hebben vaak ook een netwerk aspect en een grafische aspect. Zelfs bij technische programma's, die vaak minder grafisch intensief zijn, als matlab worden al deze aspecten verenigd. Dit drukt de noodzaak uit dat elke informaticus basiskennis heeft van computergrafiek, computernetwerken en het ontwerpen van complexe programma's. Al deze aspecten worden verenigd in dit project: \textsc{ogo} 2.3 - Nethunt. Dit document ligt ... INSERT MORE

Voor dit project werd ons gevraagd om een interactief, gedistribueerd 3D-spel te ontwikkelen. Dit houdt in dat elk speler lokaal dezelfde spelsituatie heeft, en deze op het scherm van de speler worden afgebeeld. Natuurlijk kan de afbeelding op het scherm verschillen per speler. Een andere eis was dat er voedsel aanwezig moet zijn. E\'en van de randvoorwaarden was dat elke speler ... INSERT MORE
	\newpage
    
    
    \section{Berichtsformaat}
    We zijn nu klaar om het formaat van de berichten formeel te defini\"eren. Hiervoor gebruiken we BNF (Backus-Naur Form):
    \begin{align*}
    <\text{message}> &::= <\text{command}> <\text{args}> <\text{crlf}> \\
    <\text{command}> &::= \text{``A''...``Z''} <\text{command}> | \text{``A''...``Z''} \\
    <\text{args}>    &::= ``\text{ }'' | <\text{string}> | <\text{value}> <\text{args}> \\
    <\text{string}>  &::= ``\text{ }:'' <\text{printable characters}>^{*} \\
    <\text{value}>   &::= ``\text{ }'' <\text{graphical characters except `:'}> <\text{graphical characters}>* \\
    <\text{crlf}>    &::= \text{CR LF}
    \end{align*}

    Een bericht bestaat uit een commando met argumenten. Na het commando en argumenten komt het einde van de regel: crlf. Met ``A''...``Z'' bedoelen we een willekeurige hoofdletter. Een commando is dus een hoofdletter of een hoofdletter gevolgd door een commando. Hieruit volgt dat een commando een niet-lege reeks van hoofdletters is.

    Het argument van een commando is alleen een spatie of een string of een waarde gevolgd door een argument. Een string begint altijd met een spatie gevolgd door een dubbele punt. Vervolgens volgt een willekeurige reeks van ``printable characters''. Hiermee bedoelen we letters, cijfers en spaties. Een waarde begint ook altijd met een spatie. Deze wordt altijd gevolgd door ``graphical characters''. DAARMEE BEDOELEN WE (GRAFISCH ZONDER : GEEN NUT MET DEZE DEFINITIE.

    We merken nog op dat het mogelijk is om een bericht te sturen zonder argumenten. Dit kan wel degelijk nuttig zijn, aangezien er bijvoorbeeld ook informatie kan worden gehaald uit het commando zelf.

    Het onderscheid tussen een value en een string is dus klein. Een string is altijd het laatste onderdeel van een argument. Merk op dat de verschillende argumenten worden onderscheiden door spaties. Aangezien er nog een dubbele punt voor de string staat, mag een string wel spaties bevatten zonder dat hierdoor onze berichten meerdere betekenissen krijgen. Immers, zodra we een spatie gevolgd door een dubbele punt tegenkomen, kunnen we concluderen dat we aan het laatste argument zijn begonnen. Een spatie is dan onderdeel van het argument zelf en betekent dus niet dat een nieuw argument is begonnen. Op deze manier kunnen namen van spelers en de naam van het spel spaties bevatten.

    \section{Berichten}
    We zullen nu alle gebruikte berichten afgaan met een korte toelichting. Hierbij kijken we vooral naar het nut van de ontvanger. Hiervoor gebruiken we een aantal conventies. We zullen voor de leesbaarheid komma's plaatsen tussen de verschillende argumenten. Deze zijn dus niet deel van het bericht. Ook zullen we de commando's laten eindigen met een punt, dit moet dan worden gelezen als CR LF.

    \subsection{Server detectie}
    Het \emph{Broadcast} bericht wordt gebruikt voor het detecteren van servers tijdens de initialisatie. Dit bericht word periodiek door servers verstuurd met behulp van broadcast over UDP. Hierdoor kunnen processen in hetzelfde subnet een server vinden (deze server wordt natuurlijk alleen gebruikt voor het opzetten van het spel). Het bericht bevat de versie van het spel en het huidige aantal spelers. Bovendien bevat het de naam van het spel. Het bericht heeft de volgende syntax:
    \[
    \text{GOTO, version, numPlayers :gameName.}
    \]

    \subsection{Lobby}
    Het \emph{Name} bericht wordt gebruikt na de server detectie. Als een speler de lobby van een bepaald spel wil binnengaan, wordt dit gedaan door een Name bericht te sturen. In dit bericht aan de server stuurt de speler zijn gewenste naam mee:
    \[
    \text{NAME :playername.}
    \]
    Indien de speler wordt toegelaten, zal de server een \emph{Hello} bericht terugsturen als reactie op het Name bericht (en wordt dus alleen gestuurd aan de speler die het Name bericht heeft gestuurd aan de server). Hierin staat het toegewezen ID van de speler, de versie, het aantal spelers en de naam van het spel. Dit geeft het volgende bericht:
    \[
    \text{HELLO, pid, version, numPlayers :gameName.}
    \]
    Na het versturen van het Hello bericht, zal de server ook meteen een aantal \emph{Player} berichten sturen. Dit bericht wordt aan de nieuwe speler gestuurd om het team en status van al zijn medespelers kenbaar te maken. De status kan B zijn voor niet ready, R zijn voor ready of H zijn voor host. Alle spelers zitten standaard in hetzelfde team. Bovendien hebben alle spelers standaard de status niet ready. Het spel kan pas beginnen als alle spelers ready zijn. Dan kan de speler, die de server draait, op start drukken om het spel te starten. Het bericht heeft de volgende syntax:
    \[
    \text{PLAYER, pid, tid, state :playerName.}
    \]
    Na het versturen van de Hello en Player bericht, zal de server aan alle spelers een \emph{Join} bericht sturen. Hierdoor weten alle spelers dat er een nieuwe medespeler is bijgekomen. Dit bericht wordt ook aan de nieuwe speler zelf gestuurd. Hierdoor kan de nieuwe speler concluderen dat de Player berichten zijn afgelopen. Dit bericht bevat de ID van de speler en de naam van de speler:
    \[
    \text{JOIN, pid :playerName.}
    \]
    Het \emph{Part} bericht wordt gestuurd als de verbinding van een speler met de server, al dan niet vrijwillig, wordt verbroken. Dit bericht wordt aan iedereen gestuurd om mede te delen dat een van de medespelers is vertrokken. We sturen alleen de ID van de vertrokken speler mee:
    \[
    \text{PART, pid.}
    \]
    Met een \emph{Team request} bericht kan de speler een verzoek naar de server sturen om van team te wisselen. Dit bericht bevat dan het nieuwe gewenste team van de speler:
    \[
    \text{TEAM, tid.}
    \]
    Als een speler van team is gewisseld, wordt dit kenbaar gemaakt door de server met het \emph{Team} bericht. De server stuurt dan aan de andere spelers het volgende bericht om dit kenbaar te maken:
    \[
    \text{TEAM, pid, tid.}
    \]
    Verder zijn er nog \emph{Ready request} en \emph{Busy request} berichten. Een speler kan de ready en busy berichten gebruiken om zijn status te veranderen. Ready betekent dat de status ready moet worden, terwijl Busy betekent dat de status niet ready moet worden. Dit geeft de volgende twee berichten:
    \[
    \text{READY.}
    \]
    \[
    \text{BUSY.}
    \]
    Nadat een speler van team is gewisseld, maakt de server dit kenbaar aan de andere spelers. Dit wordt gedaan door de ID van de speler mee te sturen:
    \[
    \text{READY, pid.}
    \]
    \[
    \text{BUSY, pid.}
    \]
    Een speler kan chatten met de andere spelers. Dit wordt gedaan door een \emph{Chat request} bericht te sturen naar de server. Dit bericht bevat dan het chatbericht:
    \[
    \text{CHAT :msg.}
    \]
    Zodra de server een Chat request bericht ontvangt, stuurt de server een \emph{Chat} bericht naar alle andere spelers. Dit bericht bevat de ID van de speler en het chatbericht:
    \[
    \text{CHAT, pid :msg.}
    \]
    Merk op dat er soms twee commando's zijn met dezelfde naam. Deze zijn echter zeer eenvoudig van elkaar te onderscheiden. In dit geval is er namelijk voor gezorgd dat \'e\'en commando alleen door de clients wordt verzonden en het andere commando alleen door de server. Hierdoor ontvangt elke speler nooit meer dan \'e\'en van de twee type commando's.

    \section{Opbouw van graaf}

    \section{Graaf in werking}

    \subsection{Aannamen}
    We hebben al eerder genoemd dat het \emph{Broadcast} bericht over UDP wordt gestuurd. Hier gebruiken we de broadcast functionaliteit van UDP voor. Het is dus noodzakelijk dat alle spelers zich op hetzelfde subnet bevinden. We weten dat UDP-berichten niet noodzakelijk aan hoeven te komen. We nemen echter aan dat een UDP-bericht uiteindelijk wordt ontvangen binnen een eindige hoeveelheid tijd, als de server periodiek UDP-berichten blijft versturen.

    Een speler kan dan een keuze maken tussen alle servers, waarvan een \emph{Broadcast} bericht is ontvangen. De speler gaat dan een TCP-connectie aan met de uitgekozen server. We nemen aan dat deze TCP-connectie succesvol wordt aangemaakt. Bovendien nemen we aan dat TCP inderdaad voldoet aan de FIFO-eigenschap. Verder nemen we aan dat berichten over TCP ook aankomen, eventueel nadat dit bericht een aantal keer opnieuw is verstuurd.
    
    Verder maken we nog een aantal simpele aannamen over de snelheid van het netwerk en de computers. TODO WELKE?
\end{document}
