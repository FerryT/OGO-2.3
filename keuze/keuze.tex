\section{Beschrijving alternatieven}
Tijdens het brainstormen zijn we eerst de mogelijkheden gaan afbakenen door te kijken naar een aantal genres. Hierbij hebben we vooral gekeken naar persoonlijke voorkeuren. Verder hebben we natuurlijk rekening gehouden met het feit dat het spel een `voedsel' aspect moet hebben en dat het voor meerdere spelers moet zijn. Dit gaf de volgende genres:
\begin{enumerate}
\item[i] Strategie.
\item[ii] Racen.
\item[iii] First-Person Shooter.
\item[iv] Spore.
\end{enumerate}

Hierbij merken we op dat Spore niet echt een genre is: we dachten aan een spel dat ge\"inspireerd is door het bekende spel Spore. Meerdere spelers spelen dan tegen elkaar om hun schepsel zo snel mogelijk te laten evolueren. We zullen nu elk genre dieper uitdiepen door de verschillende mogelijkheden verder in te vullen. Daarbij zullen we ook de voordelen en nadelen analyseren.

\subsection{Strategie}
Strategie was meteen een van onze persoonlijke voorkeuren. We dachten hierbij aan het winnen van terrein als een vorm van `voedsel'. Het nadeel van een terrein is dat men vastzit aan een grid. Dit is echter niet een groot probleem. Verder dachten we al snel aan RTS, real-time strategie. Dit vonden wij allen een leuk genre om te spelen. Echter, een RTS wordt al snel zeer complex om te maken, wat een groot nadeel vormt. Hierna kwamen we al snel op het idee van Tower Defence. Tower Defence is in onze mening eenvoudig, maar toch leuk om te spelen. De vraag was toen hoe Tower Defence kon worden uitgebreid tot een spel voor meerdere spelers. Het idee, wat hieruit kwam, is de basis voor het uiteindelijke spel. Een korte samenvatting hiervan is uitgewerkt in het hoofdstuk Het spel.

\subsection{Racen}
Een racespel was ons voornaamste alternatief voor strategie. Ook dit vonden wij allen een leuk genre. Bij een racespel dachten we ruwweg aan twee alternatieven. De eerste optie was om het spelconcept van Mario Kart ruwweg te volgen. De spelers rijden rondjes over een baan en kunnen power-ups verzamelen. De power-ups spelen hier dus de rol van het `voedsel'. Deze power-ups kunnen dan gebruikt worden om een voordeel te krijgen over andere spelers. Dit voordeel zou op zeer veel manieren kunnen worden gerealiseerd: zo zou een speler tijdelijk sneller kunnen gaan als gevolg van een power-up. Een andere mogelijkheid is de power-up af te kunnen schieten op andere spelers. Geraakte spelers kunnen dan worden vertraagd of tijdelijk stil komen te staan. Een mogelijk probleem is dat power-ups, die vitaal zijn om het spel leuk te maken, niet makkelijk kunnen worden ge\"implementeerd.

Een tweede optie is dat spelers rond kunnen rijden in een grote stad. Het doel is dan om de auto's van alle andere spelers te vernietigen. Een auto van een tegenstander kan worden beschadigd door tegen de zijkant aan te rijden. Er zijn twee potenti\"ele problemen bij dit spelconcept: zo is het niet duidelijk wat het 'voedsel' aspect hier is. Bovendien is het zeker niet eenvoudig om te bepalen wie tegen wie aanrijdt.

\subsection{First-Person Shooter}
De First-Person Shooter is een alom bekend spelconcept. We hebben dit spelconcept niet uitgebreid bekeken tijdens het brainstormen. Een deel van het First-Person Shooter concept is echter wel opgenomen in het uiteindelijke spelconcept bij het uitbreiden van Tower Defence. Zo kunnen spelers elkaar aanvallen in het spel. Dit wordt gedaan door elkaar te beschieten. Een van de voornaamste problemen is het bepalen of iemand is geraakt door een kogel. Aangezien dit een zeer complex spelconcept is, willen wij dit zo simpel mogelijk houden. Daarom hebben we besloten dat spelers elkaar beschieten met lasers, die een snelheid van 'oneindig' hebben. Zo kan dus direct bij het schot worden bepaald of het raak dan wel niet raak is.

\subsection{Spore}
Het welbekende Spore was voor ons ook een inspiratiebron, omdat een aantal van ons het met enthousiasme heeft gespeeld. Er is ook een zeer duidelijk `voedsel' aspect aanwezig. Het was ons echter niet duidelijk hoe dit spel op een aantrekkelijke doch eenvoudige manier kon worden uitgebreid naar een spel voor meerdere spelers.

\section{Het spel}
We geven hier slechts een korte samenvatting van het spel. Het combineert een aantal van de eerdere idee\"en, met name Tower Defence en First-Person Shooter. Een uitgebreide beschrijving van het spel staat in de spelspecificatie. We verdelen de spelers in twee teams. Spelers kunnen over het terrein rondlopen. Terwijl spelers rond lopen, kunnen ze torens bouwen met goud. Er zijn twee type torens: het eerste type toren kan op een mijn worden gebouwd en verzamelt extra goud. Het tweede type toren kan spelers aanvallen. Spelers kunnen andere spelers en torens aanvallen. Als spelers doodgaan, komen ze een tijd later weer bij het hoofdgebouw terecht. Elk team heeft ook een hoofdgebouw: het doel is dan om het hoofdgebouw van het andere team te vernietigen.

\section{Motivering keuze}